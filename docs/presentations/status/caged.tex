% TEXINPUTS=.:$HOME/git/bvtex: latexmk  -pdf <main>.tex
\documentclass[xcolor=dvipsnames]{beamer}

\input{defaults}
\input{beamer/preamble}

\setbeamertemplate{navigation symbols}{}
% \setbeamertemplate{background}[grid][step=1cm]

\usepackage{siunitx}
\usepackage{xmpmulti}
\usepackage[export]{adjustbox}

\usepackage[outline]{contour}
\usepackage{tikz}
\usetikzlibrary{shapes.geometric, arrows}
\usetikzlibrary{positioning}

\definecolor{bvtitlecolor}{rgb}{0.98, 0.92, 0.84}
\definecolor{bvoutline}{rgb}{0.1, 0.1, 0.1}

\renewcommand{\bvtitleauthor}{Brett Viren}
\renewcommand{\bvtit}{LARF 2D}
\renewcommand{\bvtitle}{\LARGE LARF Caged Gometry\\\textbf{Notes and Problems}}
\renewcommand{\bvevent}{notes}
\renewcommand{\bvbeamerbackground}{}

\newcommand{\microboone}{MicroBooNE\xspace}


\begin{document}
\input{beamer/title.tex}

\begin{frame}
  \frametitle{Caged Geometry Overview}

  New features:
  \begin{itemize}
  \item Planar (instead of finite thickness) cathode.
  \item Field cage.
  \item Bounded, fully filled wire planes.
  \end{itemize}

  Motivation for changes:
  \begin{itemize}
  \item More realistic.
  \item Make drift field more uniform.
  \item Get ready to calculate for more wires.
  \item Can use to look at edge effects.
  \end{itemize}
\end{frame}

\begin{frame}
  \begin{columns}
    \begin{column}{0.5\textwidth}
      \includegraphics[width=\textwidth]{caged-x.pdf}

      \includegraphics[width=\textwidth]{caged-z.pdf}
    \end{column}
    \begin{column}{0.5\textwidth}
      \includegraphics[width=\textwidth]{caged-iso.pdf}

      \includegraphics[width=\textwidth]{caged-y.pdf}
    \end{column}
  \end{columns}
\end{frame}
\end{document}